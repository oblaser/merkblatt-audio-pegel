%
% author        Oliver Blaser
% copyright     CC BY-SA 4.0
%

\documentclass[9pt,german]{article}
\usepackage[ngerman]{babel} % automatische Silbentrennung (deaktivieren: \mbox ; force: \-)
\usepackage[utf8]{inputenc}
\usepackage{amsmath}
\usepackage{nicefrac}
\usepackage{fontenc}
\usepackage{ccicons}

\usepackage{booktabs} % tabular
\def\arraystretch{1.2}
\newcommand{\tabitem}{~~\llap{\textbullet}~~}

\usepackage{graphicx}

\usepackage{geometry}
\geometry{
    a4paper,
    %total={160mm,247mm},
    left=25mm,
    top=25mm,
    right=25mm,
    bottom=25mm,
    %showframe,
}

\usepackage{hyperref}
\hypersetup{
    %hidelinks,
    %allcolors=black,
    linktoc=all,
    colorlinks=true,
    citecolor=black,
    filecolor=black,
    linkcolor=black,
    urlcolor=blue
}

\usepackage{biblatex}
\addbibresource{merkblatt-audio-pegel.bib}

\usepackage{array}
\newcolumntype{L}[1]{>{\raggedright\let\newline\\\arraybackslash\hspace{0pt}}b{#1}}
\newcolumntype{T}[1]{>{\raggedright\let\newline\\\arraybackslash\hspace{0pt}}m{#1}}
\newcolumntype{C}[1]{>{\centering\let\newline\\\arraybackslash\hspace{0pt}}b{#1}}
\newcolumntype{R}[1]{>{\raggedleft\let\newline\\\arraybackslash\hspace{0pt}}b{#1}}



\newcommand{\docVersion}{1.0}
\newcommand{\docTitle}{Merkblatt Audio Pegel}



\usepackage{lastpage}
\usepackage{fancyhdr}
\pagestyle{fancy}
\fancyhf{}
\fancyhead[L]{}
\renewcommand{\headrulewidth}{0pt}
\fancyfoot[L]{\footnotesize Oliver Blaser CC BY-SA 4.0 \ccbysa}
%\fancyfoot[C]{\footnotesize \thepage /\pageref{LastPage}}
\fancyfoot[R]{\footnotesize \href{https://github.com/oblaser/merkblatt-audio-pegel}{\color{black}github.com/oblaser \docTitle} v\docVersion}
\renewcommand{\footrulewidth}{0.4pt}



\renewcommand{\familydefault}{\sfdefault}
\setlength{\parindent}{0pt}



% Konvertierung zu HTML (zB zur Rechtschreibeprüfung mit Word oÄ):
% pandoc -s -o myDocument.html myDocument.tex
% Windows: %LOCALAPPDATA%\Pandoc\pandoc.exe



\begin{document}

{\Large \docTitle}

\section*{Formeln}
In der Regel wird mit Spannungen gerechnet. Somit werden folgende Formeln f\"ur die Umrechnungen zwischen
Verst\"arkungsmass $g = L_{Uout} - L_{Uin}$ und Verst\"arkungsfaktor $V=\frac{U_{out}}{U_{in}}$ verwendet:
$$ g = 20 * \log_{10}(V) \Leftrightarrow V = 10^{g/20} $$

Ein Pegel $L_U$ ergibt sich aus der Ziel- oder Messgr\"osse $U$ und dem Referenzwert $U_0$:
$$ L_U = 20 * \log_{10}(\frac{U}{U_0})  \Leftrightarrow U = U_0 * 10^{L_U/20} $$



\section*{Physikalische Bezugspegel}
\begin{tabular}{|l|l|l|}
    \hline
    \textbf{Pegeleinheit} & \textbf{Bezugs-/Referenzwert} & \textbf{Bemerkungen} \\
    \hline
    0 dBu & U\textsubscript{0} = 0.775V\textsubscript{eff} & 1mW\textsubscript{eff} an 600$\Omega$ \\
    \hline
    0 dBV & U\textsubscript{0} = 1V\textsubscript{eff} &  = 2.218dBu \\
    \hline
\end{tabular}
\\

{\small Siehe \cite[Bezugspegel]{adt} und \cite{kirstein}.}



\section*{Line Pegel}
\begin{tabular}{|c|c|}
    \hline
    \textbf{Tontechnik} & \textbf{Heimger\"ate/Consumer} \\
    \hline
    \textbf{+4dBu}              & \textbf{-10dBV} \\
    1.782dBV                    & -7.782dBu \\
    1.228V\textsubscript{eff}   & 0.3162V\textsubscript{eff} \\
    1.736V\textsubscript{P}     & 0.4472V\textsubscript{P} \\
    3.472V\textsubscript{PP}    & 0.8944V\textsubscript{PP} \\
    \hline
\end{tabular}
\\

In der Tontechnik trifft man auch Angaben f\"ur den Bezugspegel von 0dBV, das sind 1.78dB weniger als +4dBu.
Dieser Unterschied spielt meistens keine Rolle, da viele Ger\"ate einen Gain Regler haben, mit dem dieser kaum
h\"orbare Unterschied ausgeglichen werden k\"onnte.

Der Pegel -10dBV wird z.T. auch als ``Aux Pegel'' (anstatt Line Pegel) bezeichnet.

{\small Siehe \cite{kirstein} und \cite{shure}.}



\section*{Kopfh\"orer Ausgang}

F\"ur Kopfh\"orer Ausg\"ange lassen sich schlecht generelle Aussagen treffen. Ein Kopfh\"orerverst\"arker ist immer f\"ur
eine gewisse Impedanz (-Bereich) ausgelegt.
\\

Bei Mobilger\"aten scheint der maximal Pegel von 150mV\textsubscript{P} g\"angig zu sein. Smartphones erreichen
diesen Wert ggf. bereits bei ca. \nicefrac{2}{3} der Lautst\"arkeneinstellung, wobei der letzte Drittel oft eine
Warnung des Benutzers verursacht.



\printbibliography[title=Quellen]

\end{document}
